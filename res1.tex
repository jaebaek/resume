% LaTeX file for resume
% This file uses the resume document class (res.cls)

\documentclass{res}
\usepackage{color}
\usepackage{hyperref}
%\usepackage{helvetica} % uses helvetica postscript font (download helvetica.sty)
%\usepackage{newcent}   % uses new century schoolbook postscript font
\setlength{\textheight}{9.5in} % increase text height to fit on 1-page

\begin{document}

\begin{center}
\textbf{\Large Jaebaek Seo}\\
   \vspace{0.1in}
{\large Curriculum Vitae}     % the \\[12pt] adds a blank
   \vspace{0.1in}
\end{center}
                            % line after name

\address{\bf  PRESENT ADDRESS\\Room 4426 Computer Science building\\KAIST (Korea Advanced Institute of Science)\\Guseong-dong, Yuseong-gu, Daejeon, South Korea}
\address{\bf CONTACT\\Email: jaebaek@kaist.ac.kr\\Office: +82-42-350-7724\\Mobile: +82-10-2513-1216}

\begin{resume}

\section{EDUCATION}          

   Doctor of Philosophy in Computer Science at KAIST \textit{Mar 2013-Present}\\
   Master of Science in Computer Science at KAIST \textit{Feb 2011-Feb 2013}\\
   Bachelor of Science in Computer Science at KAIST \textit{Mar 2006-Aug 2010}

\section{WORKING EXPERIENCE}

   \vspace{0.1in}
   \begin{tabbing}
   \hspace{2.3in}\= \hspace{2.6in}\= \kill % set up two tab positions
    \textbf{\large Microsoft Research} \>MASS group, Intern        \>Sept 2011-Feb 2012\\
    \textbf{\large Asia} \>Beijing, China
   \end{tabbing}\vspace{-20pt}
    Led Game Sharing project for resolving the scalability problem in cloud gaming system
    in Moible And Sensor System (MASS) group.
   \begin{tabbing}
   \hspace{2.3in}\= \hspace{2.6in}\= \kill % set up two tab positions
    \textbf{\large Google Korea} \>Blogger team, Intern     \>Aug 2010-Nov 2010\\
                        \>Seoul, South Korea
   \end{tabbing}\vspace{-20pt}      % suppress blank line after tabbing
   Participated in Mobile BlogSpot project.
   It was the project for the mobile web page of BlogSpot.
   (The current mobile BlogSpot web page is the result of my project.)

\section{PUBLICATIONS}
   \vspace{0.1in}
    \textbf{FLEXDROID: Enforcing In-App Privilege Separation in Android}\\
    \underline{Jaebaek Seo}, Daehyeok Kim, Donghyun Cho, Taesoo Kim, Insik Shin,\\
    Proceedings of the 2016 Network and Distributed System Security Symposium (\emph{\textbf{NDSS}} '16),
    San Diego, CA, US, February 2016\\
    {\small(Acceptance ratio: 60/389=15.4\%)}
%   \vspace{0.1in}\\
%        \emph{Summary:} Mobile app developers often adopt third-party libraries in their apps.
%        Such integration with third-party libraries comes with the cost
%        of potential privacy violations of users, because they have same permissions
%        which their apps have.
%        To solve this problem, we suggest FLEXDROID that provides dynamic,
%        fine-grained access control for third-party libraries.
%        The key idea of FLEXDROID is to apply secure and suitable stack inspection
%        to Android.
%        Since the Android permission checkers (i.e., PM and kernel ACL checker) reside out of user apps,
%        we design a secure IPC to deliver the call stack of each user app to the permission checkers
%        (we call it \textit{Inter-process Stack Inspection}).
%        In addition, since third-party libraries can circumvent the inter-process stack inspection by JNI,
%        we sandbox JNI code of third-party libraries using the \textit{Fault Isolation} based on \textit{ARM Memory Domain}
%        (i.e., in-process memory isolation)
%        to prevent third-party libraries from counterfeiting the call stack directly or indirectly.
%        Because of the sandbox, JNI code of third-party libraries causes faults
%        when accessing shared libraries (e.g., \textsf{libc}), heap and other memory out of the sandbox.
%        Thus, we provide JNI code with independent shared libraries (e.g., \textsf{libc}), heap, stack and TLS for JNI.
%   \vspace{0.1in}\\
%        \emph{My contribution:} I engineered all system components alone.
%        The components include a character device driver for IPC,
%        Dalvik VM for handling JNI code, Android linker to support independent shared libraries,
%        independent heap (i.e., malloc), independent TLS, Android framework components related to access control and so on.

    \textbf{Optimal Real-Time Scheduling on Two-Type Heterogeneous Multicore Platforms}\\
    Hoon Sung Chwa, \underline{Jaebaek Seo}, Jinkyu Lee, Insik Shin,\\
    Proceedings of the 36th IEEE Real-Time Systems Symposium (\emph{\textbf{RTSS}} '15),
    San Antonio, Texas, US, December 2015\\
    {\small(Acceptance ratio: 34/151=22.5\%)}

    \textbf{Preventing malicious monitoring through SMS permission segmentation in Android}\\
    Eunchan Kim, \underline{Jaebaek Seo}, Byunggil Joe, Insik Shin,\\
    Korea Computer Congress, Jeju, Korea, June 2015

\section{OTHER RESEARCH EXPERIENCE}
   \vspace{0.1in}
    \textbf{Visiting student in Systems Software and Security Lab, Georgia Tech}\\
    Mar 2016-April 2016\\
    Collaboration with Taesoo Kim, Byoungyoung Lee and Ming-Wei Shih for the ASLR in SGX project

\section{SCHOLARSHIP}
   KFAS Scholarship, the Korea Foundation for Advanced Studies, 2013-2015

\section{TEACHING EXPERIENCE}
    Undergraduate Operating System course TA in KAIST from 2011 to 2015

\section{RESEARCH INTEREST}
    Systems security, mobile security, and other systems fields

%\section{AWARD}
%Chungbuk province high school student mathmatics competition $1^{st}$ winner \\
%POSTECH high school student physics competition top 30

\end{resume}
\end{document}
